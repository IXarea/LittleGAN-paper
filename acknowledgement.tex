\section*{Acknowledgement}
\addcontentsline{toc}{section}{Acknowledgement}

In this study, considering actual situation,
    we refer the opinions of teachers and choose the topic of facial image generation and adjustment.
We investigate the background togeter and jointly design model and experiments.
In this progress, Meng Yongxiang mainly writes the basic composition of the model and training code.
Chen Wenyan mainly writes part of the code and reviews the code.
Then we conduct experiments.
During the experiments, we detect, compare the result and exchange ideas to improve the model.

This paper was completed under the careful guidance of Liang Jingyun.
Her rigorous academic style has served as a model for us.

We are grateful to Professor Zheng Weishi of Sun Yat-Sen University and his team of graduate students.
He provided the necessary equipments for our experiments and provided the necessary guidance for our paper.
Finally, thank our parents for their understanding and support during our research process.

\vspace{4ex}


\textbf{Yongxiang Meng}  Student of Guangzhou No.6 Middle School.

Meng likes programming and machine learning technology and is often attend to the open source community.
His personal homepage: \url{https://github.com/ix64}.


\textbf{Wenyan Chen}  Student of Guangzhou No.6 Middle School.

Chen pays attention to technological invention, especially AI.
She has led and organized the activities of association twice.

\textbf{Jingyun Liang}  Master of Education of South China Normal University;
Senior Programmer;
Awarded the title of ‘Guangzhou Technology Innovation Expert’;

Ms Liang has been an IT teacher in Guangzhou No.6 Middle School for many years and has been experienced in instructing students in different competitions,
    such as Olympiad in Informatics,
    Scientific and Technological Innovation, Computer Works Competition.
Under her guidance, more than a dozen students have got the first prize in various national competitions,
    along with many provincial and municipal awards.


\textbf{Weishi Zheng}  Professor with Sun Yat-sen University.

His recent research interests include person re-identification, action/activity recognition, and large-scale machine learning algorithms.
He has ever joined Microsoft Research Asia Young Faculty Visiting Programme in 2015.
He is an associate editor of Pattern Recognition.
He is a recipient of the Excellent Young Scientists Fund of the National Natural Science Foundation of China,
    and a recipient of Royal Society-Newton Advanced Fellowship, United Kingdom.

